%%%%%%%%%%%%%%%%%%%%%%%%%%%%%%%%%%%%%%%%%
% Short Sectioned Assignment
% LaTeX Template
% Version 1.0 (5/5/12)
%
% This template has been downloaded from:
% http://www.LaTeXTemplates.com
%
% Original author:
% Frits Wenneker (http://www.howtotex.com)
%
% License:
% CC BY-NC-SA 3.0 (http://creativecommons.org/licenses/by-nc-sa/3.0/)
%
%%%%%%%%%%%%%%%%%%%%%%%%%%%%%%%%%%%%%%%%%

%----------------------------------------------------------------------------------------
%	PACKAGES AND OTHER DOCUMENT CONFIGURATIONS
%----------------------------------------------------------------------------------------

\documentclass[paper=a4, fontsize=11pt]{scrartcl} % A4 paper and 11pt font size
\usepackage{lmodern}
\usepackage[utf8]{inputenc} 
\usepackage[T1]{fontenc} % Use 8-bit encoding that has 256 glyphs

% \usepackage{fourier} % Use the Adobe Utopia font for the document - comment this line to return to the LaTeX default
% \usepackage[english]{babel} % English language/hyphenation
\usepackage[russian]{babel}
\usepackage{amsmath,amsfonts,amsthm} % Math packages

\usepackage{lipsum} % Used for inserting dummy 'Lorem ipsum' text into the template

\usepackage{sectsty} % Allows customizing section commands
\allsectionsfont{ \normalfont\scshape} % Make all sections centered, the default font and small caps

\usepackage{fancyhdr} % Custom headers and footers
\pagestyle{fancyplain} % Makes all pages in the document conform to the custom headers and footers
\fancyhead{} % No page header - if you want one, create it in the same way as the footers below
\fancyfoot[L]{} % Empty left footer
\fancyfoot[C]{} % Empty center footer
\fancyfoot[R]{\thepage} % Page numbering for right footer
\renewcommand{\headrulewidth}{0pt} % Remove header underlines
\renewcommand{\footrulewidth}{0pt} % Remove footer underlines
\setlength{\headheight}{13.6pt} % Customize the height of the header

\numberwithin{equation}{section} % Number equations within sections (i.e. 1.1, 1.2, 2.1, 2.2 instead of 1, 2, 3, 4)
\numberwithin{figure}{section} % Number figures within sections (i.e. 1.1, 1.2, 2.1, 2.2 instead of 1, 2, 3, 4)
\numberwithin{table}{section} % Number tables within sections (i.e. 1.1, 1.2, 2.1, 2.2 instead of 1, 2, 3, 4)

% \setlength\parindent{0pt} % Removes all indentation from paragraphs - comment this line for an assignment with lots of text
\showboxdepth = 10


%----------------------------------------------------------------------------------------
%	TITLE SECTION
%----------------------------------------------------------------------------------------

\newcommand{\horrule}[1]{\rule{\linewidth}{#1}} % Create horizontal rule command with 1 argument of height

\title{	
\normalfont \normalsize 
% \textsc{university, school or department name} \\ [25pt] % Your university, school and/or department name(s)
\horrule{0.5pt} \\[0.4cm] % Thin top horizontal rule
\huge Отчет по лабораторной работе №2 "Ручное построение
нисходящих синтаксических анализаторов" \\ % The assignment title
\horrule{2pt} \\[0.5cm] % Thick bottom horizontal rule
}

\author{Роман Голчин, M3337} % Your name

\date{\normalsize\today} % Today's date or a custom date

\begin{document}

\maketitle % Print the title



\section{Разработка грамматики}

\begin{flushleft}
Грамматика:
\begin{enumerate}
    \item$E\rightarrow E|A$
    \item$E\rightarrow A$
    \item$A\rightarrow AC$
    \item$A\rightarrow C$
    \item$C\rightarrow C*$
    \item$C\rightarrow n$
    \item$C\rightarrow (E)$
\end{enumerate}
% \end{center}

\begin{tabular}{|l|l|}
\hline
Нетерминал & Описание \\ \hline
E & Регулярное выражение\\ \hline
A& Альтернатива\\  \hline
C& Замыкание\\ \hline
\end{tabular}

\end{flushleft}

\begin{flushleft}
Грамматика содержит левую рекурсию, после ее устранения получаем:

\begin{enumerate}
    \item $E \rightarrow AE'$
    \item $E' \rightarrow | AE'$
    \item $E' \rightarrow \epsilon$
    \item $A \rightarrow CA'$
    \item $A' \rightarrow CA'$
    \item $A' \rightarrow \epsilon$
    \item $C \rightarrow nC'$
    \item $C \rightarrow(E)C'$
    \item $C' \rightarrow*C'$
    \item $C'\rightarrow \epsilon$
    
\end{enumerate}

\begin{tabular}{|l|l|}
\hline
Нетерминал & Описание \\ \hline
E & Регулярное выражение\\ \hline
E' & Продолжение регулярного выражения\\ \hline
A& Альтернатива\\  \hline
A' & Продолжение альтернативы \\  \hline
C& Замыкание\\ \hline
C'& Продолжение замыкания \\  \hline
\end{tabular}
\end{flushleft}






\section{Построение лексического анализатора}


\begin{tabular}{|l|l|}
\hline
Терминал & Токен \\ \hline
n & LETTER\\ \hline
( & LPAREN\\ \hline
) & RPAREN\\ \hline
* & ASTERISK\\ \hline
| & ALT \\ \hline
\$ & END \\ \hline
\end{tabular}



\section{Построение синтаксического анализатора}
\begin{tabular}{|l | l | l|}
\hline
Нетерминал & \texttt{FIRST} & \texttt{FOLLOW} \\ \hline
E & n, ( & \$, ) \\ \hline
E' & |, $\epsilon$ & \$, ) \\ \hline
A & n, ( & |, \$, $\epsilon$ \\ \hline
A' & $\epsilon$, n, (&|, \$, ) \\ \hline
C & n, ( & n, (, |, \$, )\\ \hline
C' & *, $\epsilon$ & n, (, |, \$, )\\ \hline
\end{tabular}

\end{document}